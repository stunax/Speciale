\clearpage
\section{Graphics Stuff}
So... Here are some graphics things which are... nifty, some you know and some you dont.

\subsection{figure import}
so your basic figure thingy
\begin{figure}[H]
	\includegraphics[scale=1]{img/swrh.jpg}
	\centering
	\caption{Fanfucking-tastic anime, haters can eat shit...}
	\label{figure:snow_white_with_red_hair}
\end{figure}
In the text afterwards, you can reference the figure by the set label. REMEMBER, to start the label with the type of thing that
is attatched to it, so for figure write something like \text{label{figure:snow\_white\_with\_red\_hair}} then if you had a table
with the same topic you can call them the same! Also, now when you are referencing, for example my super fantastic
\autoref{figure:snow_white_with_red_hair}. Use "autoref" and not just "ref", it will include the object type also, which also works with
other labels like sections, go read \autoref{section:introduction}, I dare you.

\subsection{ANIMATIONS!!!!}
OH BOY! YAYAYAYYYAYAYAYAYAYYY RAAAAAUUUULLLLL.\\
I love this shit... so we can acturally make animations in this bitch XD

You will need some gif you want to become embedded in the pdf, then, use some method or software to split
that gif into individual frames and put it in a folder. All images must have the same name but with a forth running number afterwards.
I have a unix based system, so I used "convert -coalesce name.gif name.png", and now, in the folder of the gif it will but the images
named in this case "name-x.png".
\begin{figure}[H]
	\animategraphics[loop,autoplay,width=1\linewidth,poster=7]{12}{img/gif1/emanga-}{0}{23}
	\centering
	\caption{Eromanga sensei was a masterpiece, no joke... I luved it. Everyone else are BAKAS! especially senpais...}
	\label{gif:eromanga_sensei}
\end{figure}
So, its technically a "figure", and you have a lot of different options for the "animatedgraphics" instead of "includegraphics".
Most importantly, set "loop" so that it always loops, "autoplay" so that it starts when the pdf is opened, and "poster=x" which
set the image which is displayed if the pdf is open in a program which cannot play the gif. (in this case I set it to 7)

ONE PROBLEM, only in supported pdf viewers can you see the glory which is animated image... so, at least Adope Reader.
My point is, if it enhanced the experience, just put in the gif, even if it takes up more space XD