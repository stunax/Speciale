\section{Introduction}\label{section:introduction}
There exists a hypothesis that the structure of the lumen network in pancreas, dictates which cells turn into beta cells.
To test this hypothesis, annotated data of the pancreas is required.
To annotate this data by hand would be a huge project, so we explore if it's possible to automate.

Deep learning using neural networks has become one of the main tools for computer vision.
With enough training data it is possible to use convolutional deep learning for almost every traditional computer vision task.

We study the possibility of segmenting the lumen structure of the pancreas using deeplearning.
It is tested if unlabeled data can help the prediction process, test if the prediction can benefit from preprocessing.
