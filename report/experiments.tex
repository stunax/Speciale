\section{Experiments}\label{section:experiments}
\todoyellow{Lav en intro?}
\begin{table}[H]
    \centering
    \begin{tabular}{|c|c|c|} \todored{Get correct numbers}
        Type & Trained parameters & Pretrained parameters \\ \hline
        Simple network & mange & 0 \\ \hline
        autoencoder & 4.594.949 & 0 \\ \hline
        Semi supervised & \~10.000 & \~4 mil \\ \hline
    \end{tabular}
    \caption{Amount trainable weights for each estimator}
    \label{fig:weight_matrix}
\end{table}
\subsection{Better sampling} % (fold)
\label{sub:better_sampling}
After early neural network structure testing, it was discovered that while loss looked great, the visual result did not look as good as the numbers showed.
It was decided to see if we could sample better than random, to improve the visual results of the model.

The areas that were hardest to improve was the difficult area around the tubular structure.
The experiment consists of visiually comparing images before and after sampling data, prioritizing data around the tubular structure.
As this data will be harder to predict, it is expected that the accuracy and loss will increase.
% subsection better_sampling (end)

\subsection{Simple convolution} % (fold)
\label{sub:simple_convolution}
This experiment is just to show that it is possible to achieve a result close to what is expected.
It will not have to generalize to all the data, but merely work as a start for more elaborate experiments to come.
% subsection simple_convolution (end)

\subsection{Semi supervised convolution} % (fold)
\label{sub:semi_supervised_convolution}

Because of all the trouble with the data labels, an experiment to test the gain from utilizing the unlabeled data.
It is expected that the model will generalize to unseen data, and to improve performance overall.

The idea is to pretrain a big part of the neural network using an autoencoder, and then only train the last few layers using the labels provided.

Both a visual result and a quantitive result is expected.
% subsection semi_supervised_convolution (end)

\subsection{Preprocessing} % (fold)
\label{sub:preprocessing}
The data originally have both a 3d image structure and a time axis.
The time axis will not be directly used, so an experiment utilizing the time axis for preprocessing purposes will be needed.
In addition an experiment to check if normalization will benefit prediction, is also desired.
The results are to be compared quantitively.
% subsection preprocessing (end)
