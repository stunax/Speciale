\documentclass{article}
\usepackage[utf8]{inputenc}
\usepackage [danish]{babel}
\usepackage[a4paper, hmargin={2.8cm, 2.8cm}, vmargin={2.5cm, 2.5cm}]{geometry}
\usepackage{eso-pic} % \AddToShipoutPicture
\usepackage{graphicx} % \includegraphics
\linespread{1.2}
\usepackage{amsthm}
\usepackage{amsmath}
\usepackage{url}
\usepackage{tikz}
\usepackage{amsfonts}

\author{
\Large{dpj482}\\
    \\ \texttt{}
}

\title{
  \vspace{3cm}
  \Huge{Master thesis 2018} \\
  \Large{Christian Edsberg Møllgaard}
}
\usepackage{natbib}
\usepackage{graphicx}

\begin{document}

%% Change `ku-farve` to `nat-farve` to use SCIENCE's old colors or
%% `natbio-farve` to use SCIENCE's new colors and logo.
% \AddToShipoutPicture*{\put(0,0){\includegraphics*[viewport=0 0 700 600]{natbio-farve}}}
% \AddToShipoutPicture*{\put(0,602){\includegraphics*[viewport=0 600 700 1600]{natbio-farve}}}

%% Change `ku-en` to `nat-en` to use the `Faculty of Science` header
% \AddToShipoutPicture*{\put(0,0){\includegraphics*{nat-en}}}

\clearpage\maketitle
\thispagestyle{empty}

% \newpage

\section*{Segmenting tubular structures in the pancreas using deep learning} % (fold)
\label{sec:title}
There is a biological hypotheses that during pancreatic differentiation, the
structure of the tubular lumen network (in which the cells are situated)
dictates which sorrounding cells are turning into beta cells.
Finding these structures so they can be studied, is the first step in proving
this hypothesis.
For this analysis, five 3d films of the pancreas development have been recorded on
mice, and have been annotated with some labels in preperation for this study.

The purpose of this MSc project is to segment the tubular lumen network from
the film using the deep learning library tensorflow.
% section title (end)

\section*{Project objectives and timeline} % (fold)
\label{sec:project_goal}
\begin{itemize}
    \item Explore the data, and fix errors in the dataset\\
    Deadline 15.3
    \item Segment the images using a CNN with the given set of annotated labels.\\
    Deadline 30.4
    \item Try different preprocessing methods (Normalization, filtering) and record the impact.\\
    Deadline 30.6
    \item Try different semisupervised learning techniques to include unlabeled
    data in analasys.\\
    Deadline 15.5
\end{itemize}
% section project_goal (end)

\section*{Learning goals}
\begin{itemize}
    \item Implement a CNN classifier for image segmentation using tensorflow
    \item Implement Augmented CNN in a semisupervised fashion using the unlabeled data
    \item Test improvements of segmentation when preprocessing
\end{itemize}

\section*{Risk} % (fold)
\label{sec:risk}
The data have already been shown to include some errors, so these have to be
fixed or removed before segmentation.
There is also a possibility, that more images needs annotations.
% section risk (end)

\end{document}
